\documentclass[10pt,twocolumn,letterpaper]{article}

\usepackage{report}
\usepackage{times}
\usepackage{epsfig} % You can comment this out if you use pdflatex and PDF figures
\usepackage{graphicx}
\usepackage{amsmath}
\usepackage{amssymb}

% Include other packages here, before hyperref.

% If you comment hyperref and then uncomment it, you should delete
% egpaper.aux before re-running latex.  (Or just hit 'q' on the first latex
% run, let it finish, and you should be clear).
\usepackage[pagebackref=true,breaklinks=true,letterpaper=true,colorlinks,bookmarks=false]{hyperref}


%%\reportfinalcopy % *** Uncomment this line for the final submission

\def\reportPaperID{****} % *** Enter the Project Paper ID here
\def\httilde{\mbox{\tt\raisebox{-.5ex}{\symbol{126}}}}

% Pages are numbered in submission mode, and unnumbered in camera-ready
\ifreportfinal\pagestyle{empty}\fi
\begin{document}

%%%%%%%%% TITLE
\title{\LaTeX\ Author Guidelines for Project Report}

\author{First Author\\
Institution1\\
Institution1 address\\
{\tt\small firstauthor@i1.org}
% For a paper whose authors are all at the same institution,
% omit the following lines up until the closing ``}''.
% Additional authors and addresses can be added with ``\and'',
% just like the second author.
% To save space, use either the email address or home page, not both
\and
Second Author\\
Institution2\\
First line of institution2 address\\
{\small\url{http://www.author.org/~second}}
}

\maketitle
% \thispagestyle{empty}


%%%%%%%%% BODY TEXT
\section{Layouts}

This is math in the text $sin(\alpha)$
\begin{enumerate}
\item My first Item
	\begin{enumerate}
	\item Nested item
	\end{enumerate}
\item My second item
\end{enumerate}


\section{Problem 1}

The complete strip is at most $\epsilon$.\\
Probability that we miss a strip is $1-\epsilon$.\\
Probability that N instances miss a strip is $(1-\epsilon)^n$.\\
We know that $(1-x)<=e^x$.\\

\section{Problem 2}
For a finite.

$$
\left[\begin{array}{cc}

1 & 2 \\ 
3 & 4 \\

\end{array}\right]
$$

\[
  \mathbb{E}[x] = \int_{x \in {\cal X}} x p(x) dx.
\]
$$
	[Ax]_j = \sum_{i=1}^{n}a_{j,i}x_i
$$
$$
(1-e)^n <=d
$$

%-------------------------------------------------------------------------
\subsection{References}

Example: ~\cite{Authors11}. 
Do PDFLATEX - Bibtex - PDFlatex - PDFlatex dance.

\begin{table}
\caption{Results.   Ours is better.}
\begin{center}
\begin{tabular}{|l|c|}
\hline
Method & Frobnability \\
\hline\hline
Theirs & Frumpy \\
Yours & Frobbly \\
Ours & Makes one's heart Frob\\
\hline
\end{tabular}
\end{center}
\end{table}

{\small
\bibliographystyle{ieee}
\bibliography{egbib}
}


\end{document}
